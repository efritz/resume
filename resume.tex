%% ------------------------------------------------------------------------
%% Copyright 2022 Eric Fritz (eric@eric-fritz.com).
%% ------------------------------------------------------------------------

\documentclass[]{clean-resume}

\begin{document}

\newcommand{\address}{2408 N. 89th St. Wauwatosa, WI 53226}
\newcommand{\phone}{(608) 774-1120}
\newcommand{\email}{\href{mailto:eric@eric-fritz.com?subject=Resume}{eric@eric-fritz.com}}
\newcommand{\website}{\href{https://eric-fritz.com}{eric-fritz.com}}
\newcommand{\resumelink}{\href{https://eric-fritz.com/resume/}{eric-fritz.com/resume}}
\newcommand{\articeslink}{\href{https://eric-fritz.com/articles}{eric-fritz.com/articles}}
\newcommand{\paperslink}{\href{https://eric-fritz.com/papers}{eric-fritz.com/papers}}

\header{eric fritz}{I build software}{\phone $~\cdot~$ \email $~\cdot~$ \website}

\section{core competencies}

{
  Go $\cdot$ TypeScript $\cdot$ Python $\cdot$ Postgres $\cdot$
  Containerization (Docker/K8s+Firecracker) $\cdot$
  Observability $\cdot$
  \\
  Technical writing $\cdot$ Performance optimization (algorithms, Postgres) $\cdot$
  Org-wide code standardization
}

\section{work history}

\muted{Additional work history details can be found at \underline{\resumelink}.}

\detailentry
  {Sourcegraph}
  {Staff-level Software Engineer, Lead of the Language Platform Team}
  {2019 - Now}
  {Milwaukee, WI (Remote)}
  {
    I am the primary author and owner of the data platform layer that powers precise code intelligence that the production, ingestion, persistence, and aggregation of source code index data powering code navigation features such as cross-repository go-to-definition and global find-references.
    \\\\
    My current focus within the team is to
    (1) extend the platform to enable a new set of powerful code navigation and intelligence features,
    (2) prepare the platform for the next order-of-magnitude scale, and
    (3) increase adoption by automatically indexing source code without user configuration.
    \\\\
    I often solve problems outside of my immediate team, frequently specializing in database query tuning, program optimization, and architectural/distribution concerns.
  }

\detailentry
  {Mitel}
  {Senior Software Engineer, Labs Team}
  {2015 - 2019}
  {Milwaukee, WI}
  {
    As part of the Labs team, I was the primary designer of \emph{Nighthawk}, an IFTTT-like engine and the surrounding ecosystem to support integration of internal and external services, and \emph{Kestrel}, Mitel's IoT infrastructure and collaboration strategy. Before Labs, I worked on \emph{Summit}, a CPaaS system that allows users to build voice and SMS applications with Lua code that runs in a containerized sandbox.
  }

\section{education}

\detailentry
  {Ph.D. Engineering, Computer Science}
  {University of Wisconsin - Milwaukee}
  {2018}
  {Milwaukee, WI}
  {
    \emph{`Waddle - Always-Canonical Intermediate Representation'}: an optimizing compiler and a supporting set of algorithms whose internal representation never \emph{goes stale}. Local updates to internal structures reduces compilation time while yielding the same output.
  }

\detailentry
  {M.S. Computer Science}
  {University of Wisconsin - Milwaukee}
  {2013}
  {Milwaukee, WI}
  {}

\section{publications}

\muted{An author's version of the following publications can be found at \underline{\paperslink}, and technical deep-dives about my current work are indexed at \underline{\articeslink}. Common themes include systems architecture, algorithms, optimization of Go programs, and specific advice on how to use Postgres effectively for maximum performance.}

\lineentry
  {Waddle - Always-Canonical Intermediate Representation}
  {Ph.D. Dissertation}
  {2018}

\lineentry
  {Maintaining Canonical Form After Edge Deletion}
  {ICOOOLPS}
  {2018}

\lineentry
  {Charon: The Design of a Limiting Microservice}
  {Whitepaper, Mitel}
  {2017}

\lineentry
  {Typing and Semantics of Asynchronous Arrows in JavaScript}
  {The Science of Computer Programming}
  {2017}

\lineentry
  {Arrows in Commercial Web Applications}
  {HotWeb}
  {2016}

\lineentry
  {Type Inference of Asynchronous Arrows in JavaScript}
  {REBLS}
  {2015}

\end{document}
