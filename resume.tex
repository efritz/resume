%% ------------------------------------------------------------------------
%% Copyright 2022 Eric Fritz (eric@eric-fritz.com).
%% ------------------------------------------------------------------------

\documentclass[]{clean-resume}

\begin{document}

\newcommand{\address}{2408 N. 89th St. Wauwatosa, WI 53226}
\newcommand{\phone}{(608) 774-1120}
\newcommand{\email}{\href{mailto:eric@eric-fritz.com?subject=Resume}{eric@eric-fritz.com}}
\newcommand{\website}{\href{https://eric-fritz.com}{eric-fritz.com}}
\newcommand{\articeslink}{\href{https://eric-fritz.com/articles}{eric-fritz.com/articles}}
\newcommand{\paperslink}{\href{https://eric-fritz.com/papers}{eric-fritz.com/papers}}

\header{eric fritz}{software engineer}{\phone $~\cdot~$ \email $~\cdot~$ \website}

\section{core competencies}

\shortentry
  {
    Go $\cdot$ TypeScript $\cdot$ Python $\cdot$
    Git $\cdot$ Docker $\cdot$
    Postgres $\cdot$ Redis $\cdot$
    Kafka $\cdot$ Kinesis $\cdot$ RabbitMQ \\
    Kubernetes $\cdot$ Lambda $\cdot$ DynamoDB
  }
  {}
  {}

\section{work history}

\entry
  {Sourcegraph}
  {Software Engineer on @code-intel}
  {2019 - Now}
  {Milwaukee, WI (Remote)}
  {
    I am currently exploring language servers and related offline-indexing strategies to supply fast and precise code intelligence, specifically cross-project jump-to-definition and find-reference queries, for large enterprise installations (around the scale of 100K repositories, 10M lines of code, and 1K commits per repository per day).
  }

\entry
  {Mitel}
  {Senior Software Engineer, Labs Team}
  {2015 - 2019}
  {Milwaukee, WI}
  {
    Designed and implemented: \emph{Nighthawk}, an IFTTT-like engine and the surrounding ecosystem to support integration of internal and external services; \emph{Kestrel}, Mitel's IoT infrastructure and IoT Collaboration strategy; \emph{Summit}, a CPaaS system that allows users to build voice and SMS applications with Lua code that runs in a containerized sandbox.
  }

\section{education}

\entry
  {Ph.D. Engineering, Computer Science}
  {University of Wisconsin - Milwaukee}
  {2014 - 2018}
  {Milwaukee, WI}
  {
    \emph{`Waddle - Always-Canonical Intermediate Representation'}: an optimizing compiler and a supporting set of algorithms whose internal representation never \emph{goes stale}. Local updates to internal structures reduces compilation time while yielding the same output.
  }

\entry
  {M.S. Computer Science}
  {University of Wisconsin - Milwaukee}
  {2011 - 2013}
  {Milwaukee, WI}
  {
    \emph{`Optimizing the RedPrairie Distance Cache'}: implemented and evaluated a number of caching solutions for RedPrairie's vehicular route solver using production query data given hard runtime and space constraints. Applying the chosen caching strategy provided a marked improvement in the solver's throughput.
  }

\section{publications}

Frequent technical deep-dives about my current work are listed at \articeslink. Common themes include systems architecture, algorithms, optimization of Go programs, and specific advice on how to use Postgres effectively for maximum performance.

Authors versions of the following publications can be found at \paperslink.

\shortentry
  {Waddle - Always-Canonical Intermediate Representation}
  {Dissertation}
  {2018}

\shortentry
  {Maintaining Canonical Form After Edge Deletion}
  {ICOOOLPS}
  {2018}

\shortentry
  {Charon: The Design of a Limiting Microservice}
  {Whitepaper, Mitel}
  {2017}

\shortentry
  {Typing and Semantics of Asynchronous Arrows in JavaScript}
  {The Science of Computer Programming}
  {2017}

\shortentry
  {Arrows in Commercial Web Applications}
  {HotWeb}
  {2016}

\shortentry
  {Type Inference of Asynchronous Arrows in JavaScript}
  {REBLS}
  {2015}

\end{document}
